\documentclass[a4paper, 11pt]{article}

\usepackage{graphicx}

\begin{document}

\title{Strategie Eurobot 2011}
\author{Le Mentec Fabien, IGREBOT}
\date{} 

\maketitle

\tableofcontents

\newpage
\section{Introduction et choix}
L originalite de cette annee est que les elements ne sont pas
figes sur le plateau (a l exception des zones reservees). Ils
peuvent a tout moment etre deplaces. Ce type de jeu laisse
envisager 2 classes de profil chez le robot adverse:
\begin{itemize}
\item
  positionne ses pions mais ne quittera pas sa zone de jeu.
  Si c est un bon robot, il construira des tours, surement dans les
  zones bonus. On peut envisager qu un robot cherchera a defendre
  ses elements en fin de jeu (en priorite bonus et tours), par
  exemple en refermant sa pince autour.
\item
  c est un robot plus aggressif, il penetre la zone adverse pour
  aller retirer des points a l adversaire. Cela peut s envisager
  de deux manieres:
   \begin{itemize}
   \item soit le robot pousse les pions au hasard en esperant retirer
     un maximum de points a l adversaire,
   \item soit le robot prend le temps de deplacer proprement les elements
     de jeu positionnes par l adversaire.
   \end{itemize}
\end{itemize}

A noter que les deux profils ne sont pas exclusifs, et ne sont pas
forcement ordonnes: un robot peut choisir d attaquer d abord puis
positionner ses elements de jeu ensuite.

Face a ces profils possibles, et en anticipant les capacites du
robot de cette annee, nous envisageons la solution suivante. On
decompose le match de 90 secondes en 3 phases:
\begin{itemize}
  \item la phase 1, 60 secondes: consacrees  vider la zone de distribution et
    a positionner les elments de notre camps
  \item la phase 2, 20 secondes: temps de jeu sera passe dans la zone adverse.
    le but est de retirer des points a l adversaire, et si possible de nous les
    attribuer
  \item  la phase 3: on defend la case la plus proche, si possible une zone bonus
\end{itemize}

Ce choix de decoupe du temps est motive par les raisons suivantes:
\begin{itemize}
   \item l idee, c est qu on a plus de chance de gagner un match en retirant des points
     et en se les attribuant (surtout les tours) 
   \item 90 secondes c est peu, et on prefere eviter les aller retours. Notre robot
     opere donc d abord dans zone, lentement, puis dans la zone de l adversaire,
     plus rapidement
   \item on suppose que la plupart des robots vont construire dans leur zone
     en priorite
   \item le terrain risque d etre rapidement encombre, on evite donc de revenir sur
     nos pas
   \item on suppose que nous ne pouvons pas construire de tour, on prefere prendre
     celles de l adversaire
   \item d un point de vue gestion de projet, ca permet de realiser les choses
     progressivement et dans l ordre (ie. phase 1 prioritaire)
\end{itemize}


\newpage
\section{Details sur les phases}
\subsection{Phase 1, positionnement de nos pions}
\subsubsection{Algorithme}
TODO

\subsection{Phase 2, vidage de la zone de depart}
\subsubsection{Algorithme}
TODO

\subsection{Phase 3, defense de la case la plus proche}
\subsubsection{Algorithme}
TODO

\newpage
\section{Implementation}
\subsection{Configuration du robot}
Afin d implementer cette strategie, on suppose un robot dont le modele est le suivant:
\begin{itemize}
  \item TODO: element de prehension
  \item TODO: sharps (differents modeles)
\end{itemize}

\subsection{Simulation}
Un simulateur est disponible ici: http://github.com/texane/ss

\newpage
\section{Conclusion}

\end{document}
